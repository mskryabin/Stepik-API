\chapter{Введение}
\section*{Назначение}
С помощью данного отчета можно оценить взаимодействие учащихся с учебным видео \coursetexttointro, в частности, когда  достаточно большое количество учащихся показывают аналогичные модели взаимодействия в течение короткого промежутка времени (так называемые пики взаимодействия).

\section*{Методология}

Методология для данного отчета основана на статье [Kim, J. et al., 2014]. 

Для анализа взаимодействия учащихся с видео используется два типа данных:
\begin{enumerate}
\item Повторные просмотры видео (rewatching). При анализе данных исключается первый просмотр видео. Отметим, что эта метрика не считается за уникального учащегося: если учащийся пересматривал некоторую часть видео три раза, то и метрика увеличивается на три за этот временной период.
\item Нажатие на Play (play events). При этом, игнорируется автозапуск в начале видео, так как он либо возникает автоматически, либо из-за необходимости начать смотреть видео, а не за счет выбора учащихся.
\end{enumerate}

Исходные данные о взаимодействии учащихся с видео содержит шумы. Идентификация пиков в зашумленных данных как вручную, так и автоматически становится затруднительным из-за локальных максимумов и ложных пиков. Поэтому сначала производится сглаживание данных с помощью метода LOWESS (locally weighted scatterplot smoothing) со специально подобранными параметрами сглаживания после тестирования различных значений.

После сглаживания, применяется алгоритм обнаружения пиков к обоим  типам данных: повторные просмотры видео и нажатие на Play. Используемый алгоритм является вариантом алгоритма TwitInfo [Marcus, A. et al., 2011]. Он использует взвешенное скользящее среднее и дисперсию для обнаружения необычно большого количества событий в данных временного ряда, что хорошо согласуется с данными для видео.  

Одной из причин, почему используется оба типа данных, является то, что они могут обнаружить различные модели поведения учащихся: нажатие на Play может обнаружить краткосрочные, импульсивные модели поведения, в то время как повторному просмотру соответствуют более длительные промежутки времени. В случае, если пики для повторного просмотра и для нажатия на Play перекрываются, считается, что они указывают на одно и то же событие. При этом, выбирается пик для повторного просмотра, так как он более информативный и его легче интерпретировать.

% TODO: peak features
%Особенности пика, такие как ширина, высота и площадь, может указывать на прочность коллектива, времени конкретных интересов студентов. Мы сравниваем эти функции между типами видео и студенческих контекстах. Предыдущая работа рассматриваются аналогичные конструкции в моделировании временных профилей поисковых запросов [12]. пик характеризуется описательных свойствами, как показано на рисунке 4. Она включает в себя как запуск и маркеры окончания времени, которые определяют продолжительность ширины или время пика. точка пик это самая высокая точка между [начала , конец] диапазон, который определяет высоту. И, наконец, площадь под пиком является суммой отсчетов событий во время пика временного окна, которое обозначает относительную значимость пика против всего видео.

%Несколько пиков различной профили могут появиться в видеоклипе. В отчетности высота, ширина, и область, нормировать значения путем масштабирования между 0 и 1 для решения высокой изменчивости подсчета событий и длительности через видео. Для ширины, высоты и области , мы возьмем нормированный диапазон от длительности видео, максимальное количество событий, и сумма всех отсчетов событий, соответственно.